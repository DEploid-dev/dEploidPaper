%%%%%%%%%%%%%%%%%%%%%%%%%%%%%%%%%%%%%%%%%
% Thin Formal Letter
% LaTeX Template
% Version 1.11 (8/12/12)
%
% This template has been downloaded from:
% http://www.LaTeXTemplates.com
%
% Original author:
% WikiBooks (http://en.wikibooks.org/wiki/LaTeX/Letters)
%
% License:
% CC BY-NC-SA 3.0 (http://creativecommons.org/licenses/by-nc-sa/3.0/)
%
%%%%%%%%%%%%%%%%%%%%%%%%%%%%%%%%%%%%%%%%%

%----------------------------------------------------------------------------------------
%	DOCUMENT CONFIGURATIONS
%----------------------------------------------------------------------------------------

\documentclass{letter}
%\usepackage{fullpage}
% Adjust margins for aesthetics
\addtolength{\voffset}{-0.5in}
\addtolength{\hoffset}{-0.3in}
\addtolength{\textheight}{2cm}

%\longindentation=0pt % Un-commenting this line will push the closing "Sincerely," to the left of the page

%----------------------------------------------------------------------------------------
%	YOUR NAME & ADDRESS SECTION
%----------------------------------------------------------------------------------------

\signature{Sha Zhu} % Your name for the signature at the bottom

\address{Wellcome Trust Centre for Human Genetics\\Roosevelt Drive\\Oxford OX3 7BN\\UK\\Email: joe.zhu@well.ox.ac.uk} % Your address and phone number

%----------------------------------------------------------------------------------------

\begin{document}

%----------------------------------------------------------------------------------------
%	ADDRESSEE SECTION
%----------------------------------------------------------------------------------------

\begin{letter}{ } % Name/title of the addressee

%----------------------------------------------------------------------------------------
%	LETTER CONTENT SECTION
%----------------------------------------------------------------------------------------

\opening{\textbf{Dear Editor,}}

We would like you to consider our manuscript entitled ``Deconvoluting multiple infections in {\it Plasmodium falciparum} from high throughput sequencing data'' for publication as an Original paper in {\em Bioinformatics}.

This article outlines a novel approach to deconvolute sequences of mixed samples by learning haplotype structure from a reference panel of clonal isolates. Our method is implemented with C++ as an open-source software, namely {\tt DEploid}. This program reports the number of strains, their relative proportions and their haplotypes present in an isolate, allowing researchers to study complexity of infection in malaria
with an unprecedented level of detail.

As part of the Pf3k project, an effort to map the genetic diversity of {\it Plasmodium falciparum} at global scale, we applied DEploid to deconvolute 2512 {\em P. falciparum} sequence data collected from the field, and inferred haplotypes of 4000 strains.

To our knowledge, DEploid is the first package able to deconvolute strain haplotypes and provides a unique opportunity for researchers to study inbreeding patterns and complexity of infection, leaving open the possiblity to investigate infection history at fine scale.

Appropriate reviewers for this article include:
\begin{itemize}
\item Melanie Bahlo, Population Health and Immunity, Walter and Eliza Hall Institute of Medical Research, bahlo@wehi.edu.au.
\item Daniel Lawson, School of Social and Community Medicine, University of Bristol, dan.lawson@bristol.ac.uk.
\item Daniel Falush, Department of Biology and Biochemistry, University of Bath, daniel\_falush@eva.mpg.de.
\item Kevin Galinsky, Departments of Epidemiology and Biostatistics, Harvard T.H. Chan School of Public Health, galinsky@fas.harvard.edu.
\end{itemize}


\vspace{2\parskip} % Extra whitespace for aesthetics
\closing{Sincerely,}
\vspace{2\parskip} % Extra whitespace for aesthetics

%\ps{P.S. You can find additional information attached to this letter.} % Postscript text, comment this line to remove it

%\encl{Copyright permission form} % Enclosures with the letter, comment this line to remove it

%----------------------------------------------------------------------------------------

\end{letter}

\end{document}
