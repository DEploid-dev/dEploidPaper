\input{supplementReset.tex}

\begin{center}
\textbf{\large Supplemental Materials of the Deconvolution Method Validation}
\end{center}

\section{Method validation on lab controlled strains} \label{sup:sec:validate}

The {\em P. falciparum} genetic crosses project \citep{Miles2015:sup} finds that due to sequencing error or applying different variant calling methods, genotype calls vary at the same position given the same strain of {\em P. falciparum}. Thus we apply inference methods to mutiple samples that contains the same parasite strains, and infer the genotypes of a reference strain.


\subsection{Use inference method to reconstruct the reference strains}
\begin{enumerate}
\item Mixtures of strains 3D7 and Dd2
Since 3D7 is reference strain, we can assume that strain Dd2 is the only source of `ALT' reads in samples {\tt PG0389-C}, {\tt PG0390-C}, {\tt PG0391-C}, {\tt PG0392-C}, {\tt PG0393-C} and {\tt PG0394-C}. Assume markers are independent from each other, let $y$ be the read count for `ALT' allele and $x$ be the weighted coverage, of which the weight are the proportions that are used during the mixing (see Table~\ref{tab:jason}), we use the following regression model to infer the Dd2 variant calling, $$y = \beta_0 + \beta_{Dd2} x,$$
from which significant coefficent $\beta_{Dd2}$ implies a Dd2 variant (Fig.~\ref{fig:dd2_gt1}).

\item Mixtures of strains HB3 and 7G8.
Similarly, for sample from {\tt PG0398-C} to {\tt PG0415-C},  we let variables $x_1$, $x_2$ be the weighted coverages, of which the weights are the mixing proportions for strains HB3 and 7G8 respectively. We use regression model $y = \beta_0 + \beta_{Hb3} x_1 + \beta_{7G8} x_2$ to investigate the relationships between the total allele count and weighted coverage of HB3 and 7G8. Hb3 variant is inferred as coefficients $\beta_{Hb3}$ is significant (Fig.~\ref{fig:hb3:7g8:both} and \ref{fig:hb3}), so is 7G8 (Fig.~\ref{fig:hb3:7g8:both} and \ref{fig:7g8}).
\end{enumerate}



\begin{figure}[hp]
\subfloat[]{\label{fig:dd2_gt0}
\includegraphics[width=0.5\textwidth]{validation/dd2marker709668.png}
}
\subfloat[]{\label{fig:dd2_gt1}
\includegraphics[width=0.5\textwidth]{validation/dd2marker281734.png}
}
\caption{\textcolor{red}{XXXXXXXXXXXXXX}}
\end{figure}


\begin{figure}[hp]
\subfloat[]{\label{fig:hb3:7g8:both}
\includegraphics[width=0.33\textwidth]{validation/marker73802.png}
}
\subfloat[]{\label{fig:hb3}
\includegraphics[width=0.33\textwidth]{validation/marker518472.png}
}
\subfloat[]{\label{fig:7g8}
\includegraphics[width=0.33\textwidth]{validation/marker128657.png}
}
\caption{\textcolor{red}{XXXXXXXXXXXXXX}}
\end{figure}


\subsection{Validation performance}

\subsubsection{Assessing quality of the proportion inference}
\textcolor{red}{XXXXXXXXXXXXXXXXXXXXXXXXXXXXXXXXXXX}


\begin{figure}[htp]
\centering
\subfloat[][]{
\includegraphics[width=.5\textwidth]{{subSamples/PG0402-C.subSample20.lab.errorVsTotalCoverage}.png}
}
\subfloat[][]{
\includegraphics[width=.5\textwidth]{{subSamples/PG0402-C.subSample100.lab.errorVsTotalCoverage}.png}
}\\
\subfloat[][]{
\includegraphics[width=.5\textwidth]{{subSamples/PG0406-C.subSample50.asiaAfirca.errorVsTotalCoverage}.png}
}
\subfloat[][]{
\includegraphics[width=.5\textwidth]{{subSamples/PG0406-C.subSample100.asiaAfirca.errorVsTotalCoverage}.png}
}\\
\caption{\color{red} to be done}
\end{figure}


%\begin{figure}[htp]
%\centering
%\subfloat[][]{
%\includegraphics[width=.5\textwidth]{{subSamples/PG0402-C.subSample20.asiaAfirca.errorVsTotalCoverage}.png}
%}
%\subfloat[][]{
%\includegraphics[width=.5\textwidth]{{subSamples/PG0402-C.subSample50.asiaAfirca.errorVsTotalCoverage}.png}
%}\\
%\subfloat[][]{
%\includegraphics[width=.5\textwidth]{{subSamples/PG0402-C.subSample80.asiaAfirca.errorVsTotalCoverage}.png}
%}
%\subfloat[][]{
%\includegraphics[width=.5\textwidth]{{subSamples/PG0402-C.subSample100.asiaAfirca.errorVsTotalCoverage}.png}
%}\\
%\caption{PG0402-C asiaAfrica}
%\end{figure}

%\begin{figure}[htp]
%\centering
%\subfloat[][]{
%\includegraphics[width=.5\textwidth]{{subSamples/PG0402-C.subSample20.lab.errorVsTotalCoverage}.png}
%}
%\subfloat[][]{
%\includegraphics[width=.5\textwidth]{{subSamples/PG0402-C.subSample50.lab.errorVsTotalCoverage}.png}
%}\\
%\subfloat[][]{
%\includegraphics[width=.5\textwidth]{{subSamples/PG0402-C.subSample80.lab.errorVsTotalCoverage}.png}
%}
%\subfloat[][]{
%\includegraphics[width=.5\textwidth]{{subSamples/PG0402-C.subSample100.lab.errorVsTotalCoverage}.png}
%}\\
%\caption{PG0402-C lab}
%\end{figure}

%\begin{figure}[htp]
%\centering
%\subfloat[][]{
%\includegraphics[width=.5\textwidth]{{subSamples/PG0406-C.subSample20.asiaAfirca.errorVsTotalCoverage}.png}
%}
%\subfloat[][]{
%\includegraphics[width=.5\textwidth]{{subSamples/PG0406-C.subSample50.asiaAfirca.errorVsTotalCoverage}.png}
%}\\
%\subfloat[][]{
%\includegraphics[width=.5\textwidth]{{subSamples/PG0406-C.subSample80.asiaAfirca.errorVsTotalCoverage}.png}
%}
%\subfloat[][]{
%\includegraphics[width=.5\textwidth]{{subSamples/PG0406-C.subSample100.asiaAfirca.errorVsTotalCoverage}.png}
%}\\
%\caption{PG0402-C asiaAfrica}
%\end{figure}

%\begin{figure}[htp]
%\centering
%\subfloat[][]{
%\includegraphics[width=.5\textwidth]{{subSamples/PG0406-C.subSample20.lab.errorVsTotalCoverage}.png}
%}
%\subfloat[][]{
%\includegraphics[width=.5\textwidth]{{subSamples/PG0406-C.subSample50.lab.errorVsTotalCoverage}.png}
%}\\
%\subfloat[][]{
%\includegraphics[width=.5\textwidth]{{subSamples/PG0406-C.subSample80.lab.errorVsTotalCoverage}.png}
%}
%\subfloat[][]{
%\includegraphics[width=.5\textwidth]{{subSamples/PG0406-C.subSample100.lab.errorVsTotalCoverage}.png}
%}\\
%\caption{PG0406-C asiaAfrica}
%\end{figure}







\begin{thebibliography}{}


\bibitem[\protect\citeauthoryear{Miles, Iqbal, Vauterin, Pearson, Campino,
  Theron, Gould, Mead, Drury, O{\textquoteright}Brien, Ruano~Rubio, MacInnis,
  Mwangi, Samarakoon, Ranford-Cartwright, Ferdig, Hayton, Su, Wellems, Rayner,
  McVean, and Kwiatkowski}{Miles et~al.}{2015}]{Miles2015:sup}
Miles, A., Z.~Iqbal, P.~Vauterin, R.~Pearson, S.~Campino, M.~Theron, K.~Gould,
  D.~Mead, E.~Drury, J.~O{\textquoteright}Brien, V.~Ruano~Rubio, B.~MacInnis,
  J.~Mwangi, U.~Samarakoon, L.~Ranford-Cartwright, M.~Ferdig, K.~Hayton, X.~Su,
  T.~Wellems, J.~Rayner, G.~McVean, and D.~Kwiatkowski (2015).
\newblock Genome variation and meiotic recombination in plasmodium falciparum:
  insights from deep sequencing of genetic crosses.
\newblock {\em bioRxiv\/}.

\bibitem[\protect\citeauthoryear{Wendler}{Wendler}{2015}]{Wendler2015:sup}
Wendler, J. (2015).
\newblock {\em Accessing complex genomic variation in {P}lasmodium falciparum
  natural infection}.
\newblock Ph.\ D. thesis, University of Oxford.

\end{thebibliography}
